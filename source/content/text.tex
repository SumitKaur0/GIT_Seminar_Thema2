\section{Introduction}
The ability to distinguish colors and shades of the same color plays an important role in cartography \parencite{coltekin2017}
\subsection{Choropleth maps}

\section{Basic color information}

\subsection{Human' color perception}
Although our current understanding is that color vision results from the response of three photoreceptor cells in the retina to incident light, their perception cannot be fully understood. This may be due to both individual and environmental factors that influence color perception. (Lafer-Sousa et al. 2015; Xiao et al. 2016) (Gegenfurtner and Sharpe 2001). .

Some of these factors can be, for example, the amount of light in the environment, shadows, surrounding materials, and reflectivity. In addition, the viewer's prior knowledge and cognitive biases play a significant role in color perception. (Derefeldt et al. 2004; Foster 2011). 

In addition, there is evidence that the number and distribution of photoreceptors in the eye influences what we see (Roy et al., 1991), and that our brain assumes a particular direction or light source e.g., Gegenfurtner et al. 2015; Lafer Sousa et al. 2015; Winkler et al. 2015. 

Thus, it can be said that the color perception of an individual is not stable over space and time. The same is true not only for individuals but also for groups.  

Nevertheless, there are many efforts to model and quantify color perception such as mathematical models that attempt to determine thresholds by which two colors or shades of the same color become distinguishable. 

This color distance describes a metric that quantifies the human ability to visually distinguish differences between two colors see chapter \ref{subsection:distance} \parencite{coltekin2017} 

\subsection{Color spaces}

\section{Criteria}

\subsection{Color distance}\label{subsection:distance}

\subsubsection{Equation}

\subsection{Number of classes}

\subsection{Further aspects}

\subsubsection{Spatial distance}

\subsubsection{Brightness of colors}

\section{Examples}

\section{Conclusion}


\parencite{brychtova2015}
\parencite{brychtova2017}
\parencite{sharma2005}
\parencite{coltekin2015}
\parencite{coltekin2017}

\bibliography