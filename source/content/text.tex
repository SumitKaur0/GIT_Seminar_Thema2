\section{Introduction}
The ability to distinguish colors and shades of the same color plays an important role in cartography \parencite{coltekin2017}
Lack of proper visual distance in variables colour hue and colour value is a known contributor to legibility problems in map use tasks (Chesneau, 2007; Steinrücken & Plümer, 2013; Stigmar, 2010)\parencite{brychtova2015}
\subsection{Choropleth maps}

\section{Basic color information}

\subsection{Human' color perception}
Although our current understanding is that color vision results from the response of three photoreceptor cells in the retina to incident light, their perception cannot be fully understood. This may be due to both individual and environmental factors that influence color perception. (Lafer-Sousa et al. 2015; Xiao et al. 2016) (Gegenfurtner and Sharpe 2001). .

Some of these factors can be, for example, the amount of light in the environment, shadows, surrounding materials, and reflectivity. In addition, the viewer's prior knowledge and cognitive biases play a significant role in color perception. (Derefeldt et al. 2004; Foster 2011). 

In addition, there is evidence that the number and distribution of photoreceptors in the eye influences what we see (Roy et al., 1991), and that our brain assumes a particular direction or light source e.g., Gegenfurtner et al. 2015; Lafer Sousa et al. 2015; Winkler et al. 2015. 

Thus, it can be said that the color perception of an individual is not stable over space and time. The same is true not only for individuals but also for groups.  

Nevertheless, there are many efforts to model and quantify color perception such as mathematical models that attempt to determine thresholds by which two colors or shades of the same color become distinguishable. 

This color distance describes a metric that quantifies the human ability to visually distinguish differences between two colors see chapter \ref{subsection:distance} \parencite{coltekin2017} 

\subsection{Color spaces}

\section{Criteria}

\subsection{Color distance}\label{subsection:distance}
Visual Distance in cartography is understood as a Measurement of Differences between visual variables such as size, shape, orientation and others \parencite{brychtova2015}. Here we focus on the variable of color hue and color value. The human perceived difference between two colors or color shades can be described as the color Distance. In other words, certain change of the colour in the perceptually uniform space produces equal change in human perception of that colour (Slocum et a!. 2008 \parencite{brychtova2017}).

To describe the distance of two colors scientistis have developed a method to describe the color distance. To express color quantitatively a colour space corresponding to the human perception is needed. Such color spaces are called perceptually uniform or linear. The use of such color spaces try to ensure results of color distance which are proportional to the human perception (CIE, 2012 \parencite{brychtova2015}). Presently the CIEDE2000 model ($\delta$E00, equation defined in Sharma, Wu, \& Dalal, 2005 \parencite{brychtova2015}) is regarded as the best coinciding color space with visual perception. 
 
- RGB cause RGB represents how colors are created on most digital screens. 
- RGB values do not lead to a specific color if not related to an absolut color Space
- As a color space sRGB is selected, since sRGB color space is supported by most digital screens. Other color Spaces such as Adobe RGB are not fully suported by most digital screens. When puplishing a digital map creators should choose colors which can be displayed by most screens, therefore sRGB is the color space to choose.
\subsubsection{Equation}

- transforming sRGB Colors to CIE XYZ Colors. 
- Was ist CIE XYZ Colors
- For the transformation the sRGB values has to be in the range 0.0 - 1.0, therefore the most values have to be divided by 255 to nomalize them. 
\subsection{Number of classes}

\subsection{Further aspects}

\subsubsection{Spatial distance}

\subsubsection{Brightness of colors}

\section{Examples}

\section{Conclusion}


\parencite{brychtova2015}
\parencite{brychtova2017}
\parencite{sharma2005}
\parencite{coltekin2015}
\parencite{coltekin2017}

\bibliography